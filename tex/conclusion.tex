\seclab{Conclusion}{conclusion}
From our cognitive experiment in which we tried to model how humans perceive age by looking at facial characteristics, we found some of the general characteristics that matter when humans (in this case our subjects) perceive age. Examining the results from our male experiment by looking at the right pane of \figref{M_Age}, we found that some of the facial characteristics which our subjects tended to perceive as characteristics of young males include smaller and more narrow noses, smaller probability of wearing glasses as well less noise around the mouth. On the contrary, elderly males tend to have more noise around the mouth which might be because our subjects associate beards and wrinkles as a characteristic of more mature males. Furthermore, as human noses continue to grow throughout a lifetime it makes sense that we see a correlation between bigger noses and perceived age by our subjects. Another clear trend was the fact that our subjects perceived the natural reduction of sight throughout a lifetime to be correlated with maturity and an increase in age, which can be seen by the appearance of glasses in the right pane of \figref{M_Age}. The same trends goes for the linear evolution of varying age of females in the right pane of \figref{F_Age} where our subjects again perceived a correlation between loss/reduction of sight and age, as well as with nose enlargement and age, and furthermore, found make-up and wrinkles to be correlated to maturity and age which is seen on \figref{F_Age} as noise and darker pixels around the eyes. However, the noise around the mouth typically won't resemble beard for women but rather wrinkles. \\ By assuming that we could establish a linear relationship between age and input in \secref{methodandtheory}, we restricted the model complexity a lot. Additionally by adding regularization (Lasso) as we sought a low-dimensional model, we restricted the model complexity further introducing quite a lot of bias to our model and thus reduced the variance as seen in \figref{model_complexity}. As we found that the constructed models for each gender with Lasso chose to predict around the mean, we decided to try and create a low-dimensional model with a different approach, i.e., by sequential feature selection. With the models made by sequential feature selection it was found that the models were quite similar, choosing some of the same eigenfaces to be non-zero and additionally found the same characteristics as of the first models to be important. Examples include the comments on facial hair and glasses, as well as the grainy images for women. \\ 
Moreover, we found that even though we got some interesting insights, it was hard to predict the perceived facial ages by our subjects accurately with a linear model and hence increasing the variance (model complexity) would have been appropriate if we were to make further progress towards making a model being able to accurately predict how humans perceive age based on facial characteristics. 
